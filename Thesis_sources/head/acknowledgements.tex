\chapter*{Acknowledgements}

I am grateful to my supervisor Prof. Giuseppe Foffi for the opportunity given to me to work in his group at the EPFL (and around the world!). Since the very beginning he has been very supportive and encouraging, always sharing his enthousiasm for science, giving a great freedom to explore ideas and creating an extremely friendly and familiar atmosphere in the whole group. I'm afraid I forced him to go beyond the call of his supervisoral duties at short but intense times, and I cannot but thank him for the extra patience that he exercised with me. \\
A lot of inspiration and ideas contained in this thesis come from Prof. Srikanth Sastry. I deeply thank him for giving me the oppurtunity to work in his group at JNCASR and TIFR and for devoting his energy, time and sharpness of thought in a way that went beyond a scientific collaboration. \\
Srikanth and Giuseppe both had to struggle to fight the ``Debbie-Downer-inside-me'', and I really hope that all in all it was worth it and fun.\\
I also thank Prof. Alfonso Baldereschi, for welcoming me to the ITP and introducing me to Giuseppe. I would also like to thank Noemi Porta, Tanya Castellino and Ursina Roder for all the work needed to make the ISJRP (and all the rest of the paperwork in [my] background) work. And Vittoria Rezzonico, Matteo Guglielmi and Florence Hagen for all the work related to the use of computational facilities at the EPFL. 
I thank Francesco Varrato, Nicolas Dorsaz and Alessandro Tuniz for reading the manuscript.\\
 
It will be hard to find elsewhere the same greatness that I have found in the \cancel{colleagues} friends that I have met in these four years. I really got more than what I gave, so I am indebted, and in a way that of course can't be expressed in a few sentences. The only thing I can say to excuse myself is that the standard set by them was too high. \\
Francesco has been a great teammate, and it's been fun and instructive to collaborate with him in the last months of my PhD. The word ``patience'' pops inside my mind when I think about his attitude towards me, but he's also been an example of intellectual curiosity and positive attitude during these four years. Nicolas (among the rest) gave me a lot good advices that I didn't always take, or not soon enough, because of the lack of credibility/applicability coming from the fact that he really is a bit of a hero in a lot of the things that he does (that's one of the things Giuseppe and I agree on). Maxim is a role model for clarity and scientific rigor to me, and I selfishly wish I had taken advantage of his presence more during his stay in the group. 

\clearpage
It's been fun to live in the openspace in these 4+ years and share good moments with its inhabitants: an especially grateful thought goes to the unsurpassed Old Guard: Carlo, Claude, Daniel, Paolo and Tam\'as. And to \verb|Benjamin|, for his constant encouragement and amazing puzzles I sometimes had to redshift from because of my lack of archmastery; Andrea, for her patience (with my food kleptomania, just to name one thing) and all the lifts I hitched, that turned fast into precious psychiatric help; Lorenzo, for pushing me to the swimming pool on Mondays, Salvo, for constructive criticisms about standing tall, these are difficult to accept at times! Tommaso and Frederic for (not-)being-friends-with-everybody, Duccio for making bus 22 a ``naively singular'' and fun ride and Rue de la Borde shorter and more interesting than it is, and Lucio, for making me switch from drowning people in lakes to mountaineering with great friends, just to name one reason. \\
I was very lucky to have a great company during my stay in the Sastry lab at the JNCASR and TIFR. I thank Vishwas, Aparna, Shila and Moumita for the warm welcome, their help and the fun conversations and debates (!) that we had at the JNCASR.\\
I would like to thank Sara and Olivier for hosting me in Paris during my visit to Orsay, and Alice, Susanna and Dani who made it feel like home in Zurich. Ottavia, Elisa and the summers with the Arcabalena made Switzerland good fun, so thanks!
I don't know how Anna\"ise and Letizia could cope with a flatmate of my kind, and I thank Luisa for bringing \emph{morbin} and a gust of \emph{real world} everytime we met. SilvaTania taught me French, and I hope that trying to cook mayonnaise together was fun and worth it for her too. I thank, and that's the very last groovy time, Beppe who has been a constant friendly presence, with the tales of his Consonants, all the puns and the ESP jokes, and for making me think that mine was a great mind whenever we thought alike.
I finally would like to thank my parents, Raffaella and Giuseppe, for their support and understanding, especially in the last months of completion of the thesis.

\bigskip
 
\noindent\textit{Lausanne, 13th December 2013}
\hfill Davide
