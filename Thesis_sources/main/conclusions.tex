\chapter{Conclusions and Perspectives}

In this chapter we summarize the fundamental results of this thesis. They consist in: 1) a rationalization of the observations of overaging and rejuvenation in the cyclic shear simulations reported in \cite{lacks2004energy}; 2) the analogy drawn between the behavior of our model glass and that of the sheared suspensions in \cite{corte2008random}, with the existence of a transition in the dynamics of our systems, for some value of the shear amplitude $\gamma_{c}$; 3) the existence of memory effects in our model glass, comparable to those observed in sheared suspensions in \cite{keim2011generic}; 4) the observation of a behavior that qualitatively resembles that of our model glass in the NK and TM models.\\
In what follows, we also draw connections with current research work related to ours, and describe promising future developments.\\


In the previous chapters we have studied the effect of oscillatory shear deformation on a binary mixture of Lennard-Jones particles, which we have taken as a model for metallic glasses under cyclic load. \\
By doing so, we have extended the analysis carried out on the very same system by Lacks and Osborne in \cite{lacks2004energy}, which reported a variation of the potential energy of their samples as a consequence of a \emph{single shear deformation semicycle}.
What was known from \cite{lacks2004energy} was that a single semicycle of deformation is able to overage or rejuvenate the sample depending on its initial effective temperature and on the amplitude of the deformation. Furthermore, they reported that one deformation semicycle changes samples so that they appear as having a lower or higher effective temperature (respectively in the case of overaging and rejuvenation) if one measures their energy after the deformation. However, they noted that if one measures another quantity, e.g. their stress tensor, undeformed and deformed samples are clearly distinguishable, even if they have the same potential energy. \\
What we have performed here is an analysis similar to that in \cite{lacks2004energy}, but applying a large number of full shear deformation cycles. For small deformation amplitudes, samples tend more frequently to lower their energy and overage, coherently with what is reported in \cite{lacks2004energy}. After a sufficiently large number of cycles, however, samples do assume absorbing states that are unchanged upon further cycles of the same amplitude. Observations of overaging in \cite{lacks2004energy} can thus be considered as the initial steps towards the absorbing states encountered by us. The energy of such absorbing states depends on the initial effective temperature, so that the deformation is not able to make samples lose completely the memory of their original configuration. In addition, these absorbing states have a very low residual stress, comparable to that of undeformed structures. In this case it's thus hard to distinguish deformed absorbing states and undeformed states with the same potential energy.
For larger amplitudes, systems are brought by deformation to states characterized by a potential energy and residual stress that depend on the amplitude $\gamma_{max}$ but not on the original state of the system. This is a feature that is observed only in the case of several oscillations, and was thus outside of the reach of the simulations in \cite{lacks2004energy}. A large number of shear strain cycles of sufficiently large amplitude is thus able to make systems forget about their original configuration. These states have a large residual stress, and thus can be easily distinguished from undeformed ones with the same potential energy.\\
From these observations, we conclude that there must be a threshold value of the strain amplitude $\gamma_{max}$ that determines whether a sample will lose memory of its initial configuration or not.
Incidentally, the accumulated strain $\widetilde{\gamma}_{acc}$ needed to the energy to assume a fixed value for low oscillation amplitudes and that needed to reach a steady value of the energy for large amplitudes become very large at the separation between the two regimes. 
This is the hint of the existence of a transition, which could not be observed in \cite{lacks2004energy} and is qualitatively similar to what has been observed by members of the Pine group \cite{corte2008random} in models of suspensions.\\
Our results confirm that there is indeed an analogy between the behavior of the dilute suspensions in \cite{corte2008random} and the one we observe in our model glass. In fact, by studying the diffusion of the particles in our system, we have identified two regimes, that are associated to the energy behavior and the memory of the initial conditions discussed above. For deformation amplitudes below some value $\gamma_{c}$, the systems rearrange until particles stop diffusing. This happens because diffusion makes systems explore the space of configurations until they find absorbing states which are stable under the application of a deformation cycle. Typically, the systems fall into absorbing states before having moved far away from their initial configurations, and so their initial states and the final absorbing ones are correlated.
In this low-amplitude regime one can approximate the motion of the system in configuration space as a continuous time\footnote{In our case, however, it is the accumulated strain $\gamma_{acc}$ that plays the role of time.} random walk with a finite probability of being interrupted (which we named ``mortal'' in agreement with the language used in \cite{yuste2013exploration}). This behavior is similar to what is observed in the suspensions in \cite{corte2008random}. We emphasize, however, that absorbing states in our LJ systems and particle suspensions have a different nature. In \cite{corte2008random} absorbing states are configurations where no interactions between the particles happen in the course of the deformation. In our particle model, instead, an absorbing state undergoes several rearrangements during the deformation cycle. Such rearrangements ``conspire'' to bring the system back to the initial configuration after a full cycle has been performed, so that exactly the same sequence of transitions takes place over and over again if the sample is subjected to further cycles of the same amplitude. Incidentally, an analysis of the stress-strain curves in this regime reveals that the hysteresis curves are very narrow. The energy dissipation due to the rearrangements mentioned above is thus very small.\\
For amplitudes larger than some value $\gamma_{c}$, the systems don't seem able to find absorbing states in the course of their exploration of the space of configurations, so that particle motion doesn't come to a halt. For the largest system size that we dealt with ($N=32000$) the motion of the particles is heterogeneous, with particles that move the most organizing in bands within the simulation box. To our knowledge, this is the first observation of the formation of such shear bands in simulations of oscillatory shear in Lennard-Jones systems employing Lees-Edwards boundary conditions.
In addition, the motion of the system in this regime is dissipative and characterized by large hysteresis curves in the stress-strain plot.
The value of $\gamma_{c}$ doesn't seem to be closely related to value of yield strain of our samples. Even without looking at the results in detail, this is easy to rationalize. The value of $\gamma_{c}$ is related the properties of deformed samples, and can be inferred from the behavior of oscillatory deformed samples that (above $\gamma_{c}$) have lost memory of their initial condition. There is thus no reason to believe that $\gamma_{c}$ should bear any close relation at all with the value of the yield strain, which is a property of \emph{undeformed} samples.

From all of the above it is clear that the system shows a transition from a low-amplitude regime characterized by low dissipation and the reaching of absorbing states which bear memory of the initial conditions, to a high-amplitude regime characterized by dissipation and particle diffusion and loss of memory of the initial configuration (at least for the smallest systems).
This is one of the main findings of this thesis. \\
During the completion of this work, two papers studying systems of particles under oscillatory shear were published by other authors. \\
In \cite{regev2013onset}, the transition at $\gamma_{c}$ from the regime where absorbing states dominate to the one where diffusion dominates is interpreted as a transition to chaos, with systems revisiting states not after just one cycle (as in our absorbing states), but after a small integer number of deformation cycles as the amplitude of the deformation exceeds $\gamma_{c}$. This observation is coherent with our findings and complements them. \\
In \cite{priezjev2013heterogeneous}, KA mixtures like ours are simulated at \emph{finite temperature}, and the outcome qualitatively agrees with our results: a subdiffusive plateau in the MSD of the particles is in fact observed for small deformation amplitudes, and diffusive behavior is observed for large amplitudes, thus conferming that our athermal protocol is capable of revealing features that are retained at non-zero temperatures.

The fact that both our system and the suspensions in \cite{corte2008random} both undergo a dynamic transition at some value $\gamma_{c}$ of the oscillation amplitude suggests that the two systems can show other similarities. 
We indeed verify that our Lennard-Jones system can retain a memory of its mechanical history like suspensions studied in \cite{keim2011generic} do. In particular, if a set of LJ systems is trained with cyclic oscillations of a given amplitude, such an amplitude can be read by performing a single ``reading'' cycle of various amplitudes. The memory manifests itself as a kink in the MSD of the samples at the end of each reading cycle. Differently from what is observed in the case of suspensions, samples trained with a given amplitude reach states that are absorbing only for cycles whose amplitude coincides exactly with the training amplitude, but are destabilized by cycles of lower or higher amplitude. An indirect consequence of this fact is that it is possible to encode multiple memories in our samples by applying on the sample an arbitrarily long series of training oscillations alternating different amplitudes. This is not possible in the noise-free version of the system in \cite{keim2011generic}, because in that case a long training has the consequence of erasing all but one memory of the training amplitudes. Our system, which is able to exhibit multiple \emph{persistent} memories, is instead equivalent to the system in \cite{keim2011generic} with added noise. 

We expect the results above to extend to more complex systems (for instance constituted by a larger number of components) endowed with more realistic interaction potentials. In addition, our work can be considered as a benchmark for mesoscopic theories of deformation (e.g. the STZ theory). Faithful mesoscopic descriptions should thus aim at reproducing, for instance, the transition behavior described above. 

In the previous chapter we have also shown that the behavior observed in oscillatory deformation of our LJ system can be observed in the NK model, a model which has been used in the past \cite{isner2006generic} to show that rejuvenation and overaging don't occur only in particle models of glasses. Our results extend the observations on the NK model in \cite{isner2006generic} in a similar way as we have extended the work on the LJ system in \cite{lacks2004energy}. Our analysis of the NK model demonstrates that a transition and memory phenomena can be observed in presence of a generic deformable landscape. This fact suggests, for instance, that memory effects can be observed in driven systems where the dynamics of the system can be considered athermal and occurring in a deformed landscape. This could be realized, for instance, in the case of frustrated magnets at low temperature in oscillating magnetic fields.
Incidentally, the NK model appears to as an ideal playground to test the existence of a mechanism of ``period-doubling'' for the transition to chaos explored in \cite{regev2013onset} in athermally deformed energy landscapes.

We have also developed and described in detail the TM model, which is a toy model that aims at modeling in a very simple way the athermal dynamics of the systems above, based on some fairly stringent assumptions on the behavior of the energy landscape under deformation. In spite of its simplicity (and in particular the absence of a notion of distance between the inherent structures) it has proven able to reproduce the existence of some kind of transition at a value $\gamma_{c}$ and to show memory effects. The very limited number of ingredients in this model leads us to believe that the phenomenology observed for the LJ and NK systems can be observed in a wide class of systems showing some kind of ``symmetry'' in their dynamics, which is embodied in the construction on $P_{\pm}$ matrices in the TM model.
